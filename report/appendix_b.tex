\section{Appendix B - Microcontroller VHDL}
\subsection{Top level module}
\lstinputlisting[language={vhdl}, caption={Top level module for the
microcontroller},
label={lst:uContTop}]{../ucontrollertest/src/basys_shell.vhd}

\subsection{Input synchronizer}
\lstinputlisting[language={vhdl}, caption={VHDL code for the input
synchronizer of the microcontroller},
label={lst:uContSync}]{../ucontrollertest/src/InputSynchronizer.vhd}

\subsection{Operations}
\lstinputlisting[language={vhdl}, caption={VHDL code for the operations entity
of the microcontroller}, label={lst:uContOp}]{../ucontrollertest/src/operations.vhd}

\subsection{RAM}
\lstinputlisting[language={vhdl}, caption={VHDL code for the RAM entity of
the microcontroller}, label={lst:uContRAM}]{../ucontrollertest/src/ram.vhd}

\subsection{Reg}
\lstinputlisting[language={vhdl}, caption={VHDL code for the Reg entity of
the microcontroller}, label={lst:uContReg}]{../ucontrollertest/src/reg.vhd}

\subsection{ROM}
\lstinputlisting[language={vhdl}, caption={VHDL code for the ROM entity of
the microcontroller. Note that this is without any actual program, as that
would just be binary nonsens. Instead ASM code will be included for all
programs}, label={lst:uContROM}]{../ucontrollertest/src/rom_for_show.vhd}

\subsection{Vending machine code}
\lstinputlisting[caption={},
label={lst:vendingCode}]{../ucontrollertest/compiler/ColaDone.txt}
