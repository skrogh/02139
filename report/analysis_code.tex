\section{Coding the microcontroller}
Since this was a cource in hardware dessign and not software dessign, this
section will be rather brief.

For the fun of it we decided to impliment the vending mashine with a lot of
extra features. We also changed the specifications for this microcontroller
vending machine a bit.
\begin{itemize}
  \item Added software debounce of the toggles.
  \item The buttons are no longer used for coin and user input,
  instead the three leftmost toggles serve this purpose. This was done to demo
  the debounce.
  \item Price and entered coins are no longer displayed side by side, but one at
  a time, blinking. A \emph{t} predecessing the number indicates total, a
  \emph{p} price.
  \item Alert and dispense are no longer displayed by green LEDs but scrolling
  messages.
  \item The price can still be set via switches.
  \item The vending machine will freeze and display \emph{empty} after 16 cans
  of soda.
\end{itemize}

\subsection{Analysis and design}
The vending machine program contains a main loop,
resposible for display multiplexing and debouncing buttons:

The content of $r12-15$ will be used to lookup in the RAM for a number/letter to
be displayed on the four displays.
The debounced button state is kept in one place of the RAM,
while a variable, ticking high for one loop on a rising edge from a button is
kept another place in the RAM.

once the main loop finishes it returns to one of the action loops,
there are four of these:
\begin{itemize}
  \item getCoinsLoop/wait: This is where the vending machine starts.
  The price is updated and if the machine is empty you will go to that loop.
   Entering coins will add to total. pressing buy will send you to either one of the two
  following states. The screen will alternate between showing the price and the
  total.
  \item dispenseLoop: This is were you end during purchase of a can, if you can
  afford it, once buy is lifted you return to getCoinsLoop.
  \item alertLoop: This is were you end during purchase of a can, if you can't
  afford it, once buy is lifted you return to getCoinsLoop.
  \item emptyLoop: This is where you end if the machine is empty, you can not
  get away from here without a reset.
\end{itemize}
each loop puts some value in the $r12-15$ registers to be shown to the screen.

A flowchart over the program can be found in figure~\ref{fig:progFlow}
\stdfig{0.8}{programFlow}{Flowchart over the program for the vending machine
microcontroller}{fig:progFlow}
