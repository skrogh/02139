\section{Discussion}
The two different ways of simulating the vending machine both have their pros and cons.
The FSMd implementation is very optimal since it is designed with a specific implementation in mind.
This results in a minimum amount of VHDL and testing required to verify that the design performs as
expected. However, if a new feature is required, the design process must often be started from scratch.
This was the reason for implementing a programmable microcontroller. Even though the initial design
burden was fairly large, when the initial design was implemented and the compiler was up and running.
The speed at which new features could be added was much greater than the FSMd approach, however the resource
consumption and amount of testing necessary to verify the design was much greater. \\

If the design platform and intended application  is taken into account, an FSMd implementation is most likely the optimal choice.
This is because an FPGA can be reprogrammed so easily, that the cost of changing the FSMd design is
practically zero apart from the time spent on adding features to the design and because the vending machine application is fairly
fixed in it's requirements.\\

The microcontroller can however be used for different applications than the one outlined in this report, with a simple
reprogramming that does not change the hardware implementation, but only the executed code. This means that the amount
of hardware testing down the line is neglible, but the software will require testing, which is also costly. \\

Thus we can, not surprisingly, conclude that for a fixed application where design time and monetary resources are not
an issue, but physical resources are, an FSMd is the optimal design choice. On the other hand, when the application
does not have fixed requirements or there are time or money constraints, but physical resources are not an issue, 
a microcontroller might be more cost effective.
