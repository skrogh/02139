\subsection{Implementation}
This section contains the implementation details regarding the FSMd implemenmation of the vending machine controller all the listings referred to in this section can be found in appendix A.
Again, this will not cover the inputsyncronizer and clock manager since these are already provided. \\

The top level module can be seen in listing \ref{fsmd_top}.
\subsubsection{Display Driver}
All of the VHDL code relating to the FSMd implementation of the vending machine controller can be found in Appendix A. \\


The VHDL implementation of display driver design seen in figure \ref{display_sch} is found in listing 
\ref{display_top}.
This VHDL code shows the connections between the BINARY TO BCD modules and the segment multiplexer entity, that covers both the FSM and multiplexing.
The top level display entity does not contain the HexToSevenSegment entity, since this is connected directly to the segment multiplexer.

\subsubsection{Vending Machine CPU}
The VHDL code describing the vending machine CPU can be found in listing \ref{CCPU}.
Two architectures have been created, one that directly translates the ASM chart in figure \ref{cola_asm}, and one that realises the FSM and datapath in figure \ref{cola_fsm} and \ref{fsmd_datapath} respectively.


\subsubsection{Resource Consumption}
The resource consumption of this design is somewhat high compared to the functionality of the controller.
When synthesized, the design takes up 125 LUTs on the FPGA according to the synthesis report which can also be found in appendix A.
However, must of these resources go to the lookup tables for binary to BCD conversion, and the clock management.
