\section{Microcontroller Implementation}
Instead of using the FSMd approach to add more than the basic features to the vending machine controller,
we decided to build a simple programmable softcore microcontroller to avoid having to modify the datapath
and FSM every time we wanted to add a feature. We wanted the microcontroller to be able to run
on the Basys2 board, but not be limited to it. Thus some design decisions have been made that
may seem strange for the Basys2 board.
This section covers how the microcontroller was designed
and implemented. \\

In the end a working vending machine program for the microcontroller is presented.

\subsection{Analysis and Design}
The microcontroller in it's finished state must be able to run a program, that simulates a vending machine, just like the FSMd
implementation of the vending machine controller. Therefore it must have enough output and input pins to control the
LEDs, and take input from the toggleswitches and pushbuttons on the Basys2 board. It must also be fast enough 
to multiplex the 7-segment display, since we did not want to have an external display driver in parallel, thus making
the ucontroller as pure as possible. The onboard 50MHz clock was deemed fast enough for this. This leaves a 
minimum I/O specification of
\begin{itemize}
    \item 20 output pins, 8 segment cathodes, 4 segment anodes, 8 LEDs.
    \item 12 input pins, 4 debounced pushbuttons, 8 toggle switches.
\end{itemize}
Because of simplicity we decided to let the CPU at the heart of the microcontroller be an 8-bit CPU. This also fits well with limited
amount of memory on the Basys2 board. To further minimize the amount of memory used, the CPU has only 16 internal
registers r0 through r15, addressable with 4 bits, and a limit of 32 opcodes, which can be encoded in 5 bits. Letting all the
registers double as accumulators also saves space. This means that
instructions that operate on 2 registers, need 8-bit for the register addressing and 5 bit for the opcode, totaling 13 bits. 
This instruction size allows for a sizeable amount of program code, 5000+ instructions on the Spartan-3E residing on the
Basys2 board. \\

To simplify the logic needed to decode the instructions, every instruction consists of, in order, 5 opcode bits,
4 destination register bits and 4 source register bits. From here on, opcodes will be denoted as opc, destination registers as
rd/Rd and source registers as rs/Rs. \\

For I/O, 16 registers are available, the first eight of which are input registers and the last eight output registers.
These can be set or read using special opcodes.\\

The microcontroller also includes 256 bytes of ram, which can be accessed with associated opcodes.

\subsubsection{Opcodes}
The instruction set for the processor is seen below together with its assembly mnemonics. It was created
with a focus being simple, while still having the capability of doing most common operations
in a single clock cycle. All operations shown below are single clock cycle operations.
%spec listing
\lstinputlisting[caption={Instruction set for the soft core microcontroller}, label={spec}]{clean_spec.txt} 
