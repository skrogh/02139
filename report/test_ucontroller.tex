\subsection{Test}
A small testing program going through all operations is written for the
micrcontroller. Each line also describes the desired output and contains the
line number in hexcode for easy verification in the figures.


\lstinputlisting[caption={Assembly code for the testing of the microprocessor,
}, label={lst:uContTestCode}]{../ucontrollertest/compiler/OperationTest.txt}

In figure~\ref{fig:uContTest1} and figure~\ref{fig:uContTest2} the result of the
test is shown, the easiest way to check, that everything works as expected is to
cross check the assembly-code comments and the figures. Note that the processor
is reset during code execution to show that the RAM maintains its values.

\stdfig{1}{uControllerTestP1}{The microprocessor being testet
with the code from listin~\ref{lst:uContTestCode}. Note that not all signals
are shown, as there are way too many. Page 1/2}{fig:uContTest1}
\stdfig{1}{uControllerTestP2}{The microprocessor being testet
with the code from listin~\ref{lst:uContTestCode}. Note that not all signals
are shown, as there are way too many. Page 2/2}{fig:uContTest2}

A minor bug was found during testing. Code line 0 is skipped, and line 1
repeated for as long as reset is held. Looking at the \emph{VHDL} it is quite
apparent why this is so, it would not be hard to rewrite, but for now the fix is
simply to use the \emph{nop} command on line 0.

The do file used during testing can be found in listing~\ref{lst:uContTestDo}

\begin{lstlisting}[caption={Do file for the test of the microcontroller},
label={lst:uContTestDo}]
restart
force reset_async 1 0, 0 4, 1 120, 0 122
force clk 0 0, 1 1 -repeat 2
force I_0_async 0 0, 8 20
force I_1_async 0
run 130
\end{lstlisting}