\subsection{Test}
Testing of the FSMd implementation of the vending machine controller is done with modelsim.
In order to insure that the design is working as intended, it should be verified that
all of the requirements from the specification are met. 
This can be done with the following test sequence.
\begin{enumerate}
    \item Assert \textbf{coin2}, verify that the controller adds 2 to \textbf{sum} and waits for \textbf{coin2} to be deasserted.
    \item Assert \textbf{coin5}, verify that the controller adds 5 to \textbf{sum} and waits for \textbf{coin5} to be deasserted.
    \item Assert \textbf{buy}, verify that the controller can compare \textbf{sum} and \textbf{price} correctly and assert either \textbf{alarm} or \textbf{release\_can}.
    \item Assert \textbf{reset}, verify that the controller resets it's state and \textbf{sum}.
\end{enumerate}
It can also be verified that the controller shows the correct numbers on the 7-segment display using modelsim, however this is much easier to do 
visually using the Basys2 board. \\

Since operation of the machine assumes that the \textbf{sum} will never exceed the maximum value the machine can handle, this will not be tested.
Should it happen, it would cause overflow because of the datapath implementation.
